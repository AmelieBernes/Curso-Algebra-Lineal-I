\section{Simulacro de examen, primer parcial}

\marginnote{
Estos fueron
los problemas del examen parcial aplicado al curso del Otoño 2024.
Puedes contestar todos los que quieras en dos horas, y dependiendo de qué
tan bien está tu solución, podrás ganar una fracción de los puntos que vale
cada problema. Si acumulas 10 o más puntos, tu calificación es 10.
}
\begin{prob}
(0.5 p.)
En un $F-$espacio vectorial, ¿el vector cero
puede ser elemento de un subconjunto
linealmente independiente?
Argumenta tu respuesta.
\end{prob}

\begin{prob}
(0.5 p.) Sean $V$ un $F-$espacio
vectorial, $n \geq 1$ un entero. ¿Qué significa
que la dimensión de $V$ sea $n$?
\end{prob}

\begin{prob}
(0.5 p.)
Explica por qué el $\IR-$espacio vectorial de los
polinomios con coeficientes reales
es infinito dimensional.
\end{prob}

\begin{prob}
(0.5 p.) ``Si $X$ genera 
al $F-$espacio vectorial $V$, entonces cualquier vector de $V$
puede expresarse de forma única como combinación lineal
de elementos de $X$''.
¿Es esto cierto o falso?
Argumenta tu respuesta.
\end{prob}

\begin{prob}
(1 p.) Demuestra que, en 
un $F-$espacio vectorial $V$, un singulete
$\{ x \}$ es linealmente
independiente si y sólo
si $x \neq 0$.
\end{prob}

\begin{prob}
(1 p.) Sea $V$
un $\IZ_{2}$ espacio vectorial. Demuestre que
todo vector coincide con su inverso aditivo,
es decir, que
\[
\forall v \in V: \hspace{0.2cm}
v = -v.
\]
\end{prob}

\begin{prob}
(1 p.)
Sea $X \neq \emptyset$ un conjunto no vacío,
$F$ un campo. Considere al $F-$espacio
vectorial $F^{X}$ de las funciones de 
$X$ en $F$. Seleccionando un punto
$x_{0} \in X$, defina
\[
W:= \{ f \in F^{X}  | \hspace{0.2cm} 
f(x_{0}) = 0 \}.
\]
Demuestre que $W$ es un subespacio de
$F^{X}$.
\end{prob}

\begin{prob}
(2 p.)
Sea 
\[
U = \{ (a, b, c) \in \IR^{3}  | \hspace{0.2cm}
a^{2} = c^{2}  \}.
\]
¿Es $U$ un subespacio del $\IR-$espacio vectorial
$\IR^{3}$? Argumenta tu respuesta.
\end{prob}

\begin{prob}
(3 p.)
Sea $V$ un $F-$espacio vectorial. Sea la
familia
\[
\mathcal{I} = \{ X \subseteq V | \hspace{0.2cm}  X \textit{ es linealmente independiente} \}.
\]
Argumente si las siguientes proposiciones son
verdaderas o falsas:
\begin{itemize}
	\item La familia $\mathcal{I}$ de subconjuntos de $V$
	es no vacía.
	\item Si $X \in \mathcal{I}$, entonces
	todo subconjunto de $X$ es también
	elemento de $\mathcal{I}$.
	\item Si $X, Y \in \mathcal{I}$, 
	entonces $X \cup Y \in \mathcal{I}$.
\end{itemize}
\end{prob}

\begin{prob}
	\label{prob:10 de examen 1}
(3 p.) Considere al $\IR-$espacio vectorial
$\IR^{3}$. Si 
\[
X = \{ v_{1} = (1, 1, 1), v_{2}
= (0, 1, -1), v_{3}= (1, 0, 2) \}.
\]
Demuestra que el vector $u = (4, 2, 1) \in \IR^{3}$ \textbf{no}
es elemento del generado de $X$.
\end{prob}

\begin{prob}
(3 p.) del problema \eqref{prob:10 de examen 1}
se infiere que el conjunto $X$
(que tiene $3$ elementos) no genera a 
$\IR^{3}$ (siendo este un espacio de dimensión
$3$), por lo tanto, no puede ser
linealmente independiente.
Quita un vector $v$ de $X$ de tal forma que
$X - \{ v \}$ sea linealmente independiente,
y extiende este subconjunto linealmente
independiente a una base
de $\IR^{3}$
\end{prob}

\begin{prob}
(3 p.) El conjunto
\[
\beta = 
\left\{
A = 
\begin{pmatrix}
0 & 1 \\
2 & 3
\end{pmatrix}, \hspace{0.1cm}
B = \begin{pmatrix}
1 & 12 \\
14 & 8
\end{pmatrix}, \hspace{0.1cm}
C = \begin{pmatrix}
1 & 9 \\
8 & -1
\end{pmatrix}, \hspace{0.1cm}
D = \begin{pmatrix}
1 & 7 \\
6 & 2
\end{pmatrix}, \hspace{0.1cm}
E = \begin{pmatrix}
1 & 6 \\
8 & 21
\end{pmatrix}, \hspace{0.1cm}
F = \begin{pmatrix}
1 & 5 \\
4 & 7
\end{pmatrix}
\right\}
\]
genera al $\IR-$espacio vectorial 
$M_{2 \times 2}(\IR)$. De este
generdor extrae una base para el 
espacio $M_{2 \times 2}(\IR)$.
\end{prob}
