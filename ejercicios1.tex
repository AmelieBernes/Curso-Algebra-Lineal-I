\section{Ejercicios I}
	\label{section: ejercicios I}
En lo que sigue, a menos que se indique lo contrario,
$V$ es un $F-$espacio vectorial.
Los ejercicios marcados con el símbolo
$\mathbat$ son obligatorios para los estudiantes de matemáticas,
pero opcionales para los de actuaría.

\begin{ej}
    Demuestre que
    \[
    W_{1} = \{ (a_{1}, \ldots , a_{n}) \in \IR^{n} : \hspace{0.2cm} 
    a_{1} + \cdots + a_{n} = 0\}
    \]
    es un subespacio de $\IR^{n}$, pero que 
    \[
    W_{1} = \{ (a_{1}, \ldots , a_{n}) \in \IR^{n} : \hspace{0.2cm} 
    a_{1} + \cdots + a_{n} = 1\}
    \]
    no lo es.
\end{ej}

\begin{ej}
    Demuestre que un subconjunto $W$ de un espacio vectorial $V$
    es un subespacio de $V$ si y sólo si
    \begin{enumerate}
        \item $0 \in W$, y
        \item para todo $a \in F$ y $x, y \in W$, se tiene que
        $ax + y \in W$.
    \end{enumerate}
\end{ej}

\begin{ej}
\hlpink{\textbf{El $F-$espacio vectorial de polinomios con coeficientes en
un campo $F$:}}
Sea $F$ un campo. Un \textbf{polinomio con coeficientes en el campo $F$}
es una función $f : F \longrightarrow F$ de la forma
\begin{align}
	\label{eq: pol f}
f(x) = & a_{n}x^{n} + a_{n-1}x^{n-1} + \cdots + a_{1}x + a_{0} \nonumber \\
= & \sum_{i=0}^{n} a_{i}x^{i},  
\end{align}
\marginnote{Haciendo $F = \IR$, el polinomio
$f(x) = x^{2}-1$ tiene grado dos y es mónico, $f(x) = 4$
tiene grado cero.}
donde $n \geq 1$ entero y $a_{0}, a_{1}, \ldots , a_{n} \in F$. 
A los $a_{i}$ se les conoce como
\textbf{coeficientes del polinomio}. Si todos los coeficientes
son cero, decimos que $f$ es el \textbf{polinomio cero}, 
y definimos a su \textbf{grado} como $-1$.
Si $a_{n} \neq 0$, decimos que el grado del polinomio
es $n$ y $a_{n}$ es su \textbf{coeficiente principal}.
Todo polinomio cuyo coeficiente principal sea $1$ será llamado
\textbf{mónico}.

Dos polinomios $f(x)$ y $g(x)$ son iguales si y sólo si 
tienen el mismo grado y sus correspondientes coeficientes
son iguales.

Sea
\[
F[x] := \{ f \in F^{F}  | \hspace{0.2cm} f \textit{ es un polinomio} \}
\]
el conjunto de todos los polinomios con coeficientes en $F$.
Definimos 
\marginnote{
Puedes encontrar esta definición
del espacio vectorial de polinomios en el Friedberg, p.10. 
Te recomiendo que estudies también cómo $F[x]$ es 
esencialmente igual al espacio $F^{(\IN)}$ de sucesiones
en $F$ con soporte finito.}
\begin{enumerate}
	\item La suma de dos polinomios
	\[
	f(x) = \sum_{i=0}^{n} a_{i}x^{i},
	\hspace{0.2cm}
	g(x) = \sum_{i=0}^{m} b_{i}x^{i},
	\hspace{0.3cm} a_{n}, b_{m} \neq 0
	\]
	como sigue; sin pérdida de generalidad, supongamos que
	$n \leq m$. La suma de $f$ con $g$ es
	\[
	(f+g)(x) = \sum_{i=0}^{m} (a_{i} + b_{i}) x^{i},
	\]
	donde, para $n \leq i \leq m$, definimos $a_{i} = 0$.
	
	\item La multiplicación escalar de un polinomio cualquiera
	\eqref{eq: pol f} por un escalar $\alpha \in F$ como
	el polinomio
	\[
	(\alpha f )(x) = \sum_{i=0}^{n} (\alpha a_{i}) x^{i}
	\]
\end{enumerate}


\begin{itemize}
	\item Demuestre que $F[x]$ es, con la suma y la multiplicación
	escalar definida arriba, un $F$-espacio vectorial.
	\item Sea
	\[
	M[x] = \{ f \in \IR[x] : \hspace{0.2cm} f \textit{ es mónico} \}.
	\]
	¿Es $M[x] \leq \IR[x]$?
	\item ¿Es $\IR[x] - M[x] \leq \IR[x]$?
	\item Para $n \geq 0$, Sea
	\[
	\IR_{n}[x] = \{ f \in \IR[x]  | \hspace{0.2cm} \textit{el grado de f es 
	menor o igual a n}  \}.
	\]
	Demuestre que $\IR_{n}[x] \leq \IR[x]$. Muestre además que
	$\{ 1, x, x^{2}, \ldots , x^{n} \}$ genera a $\IR_{n}[x]$.
\end{itemize}
\end{ej}

\begin{ej}
\hlpink{\textbf{El $F-$espacio vectorial de matrices de dimensión $m \times n$
con coeficientes en $F$}}: 

Dado un campo $F$, una \textbf{matriz 
$m \times n$ dimensional con coeficientes en $F$}
es una función de la forma
\[
A : \{ 1, \ldots, m \} \times 
\{ 1, \ldots, n \} \longrightarrow F.
\]
Normalmente abreviamos
\[
A(i, j) = a_{i,j},
\hspace{0.2cm} 1 \leq i \leq m, \hspace{0.1cm}
1 \leq j \leq n,
\]
y representamos a la matriz $A$ como un arreglo rectangular
\[
A = (a_{ij}) =
\begin{pmatrix}
a_{11} & a_{12} & \cdots & a_{1n} \\ 
a_{21} & a_{22} & \cdots & a_{2n} \\
\vdots & \vdots & \vdots & \vdots \\
a_{m1} & a_{m2} & \cdots & a_{mn}
\end{pmatrix}.
\]
Definimos tanto la suma de matrices como la multiplicación
escalar entrada a entrada: para cualesquiera
$A = (a_{ij}), B = (b_{ij})$, $\alpha \in F$,
\[
A + B := (a_{ij} + b_{ij}), 
\hspace{0.2cm}
\alpha A := (\alpha a_{ij}).
\]

\begin{itemize}
	\item Demuestre que el conjunto $M_{m \times n}(F)$
	de las matrices $m \times n$ dimensionales con entradas en 
	$F$ y las operaciones definidas arriba es un $F-$espacio vectorial.
	\item Demuestre que las matrices 
	$$\begin{pmatrix}
1 & 0 \\
0 & 0 
\end{pmatrix},
\hspace{0.1cm}
\begin{pmatrix}
0 & 1 \\
0 & 0 
\end{pmatrix},
\hspace{0.1cm}
\begin{pmatrix}
0 & 0 \\
1 & 0 
\end{pmatrix},
\hspace{0.1cm}
\begin{pmatrix}
0 & 0 \\
0 & 1 
\end{pmatrix}
$$
generan a $M_{2 \times 2}(F)$.
	\item Demuestre que
	\[
	D_{n} := \{ A=(a_{ij}) \in M_{n \times n}(F)  | \hspace{0.2cm}
	i \neq j \Rightarrow a_{ij} = 0  \},
	\]
	el conjunto de las matrices diagonales $n-$dimensionales,
	es un subespacio de $M_{n \times n}(F)$.
	\item Si 
	\[
	TS_{m,n} := \{ A=(a_{ij}) \in M_{m \times n}(F)  | \hspace{0.2cm} 
	i > j \Rightarrow a_{ij} = 0 \}
	\]
	es el conjunto de las matrices triangulares superiores
	de dimensión $m \times n$, demuestre que
	$TS_{m,n} \leq M_{m \times n}(F)$.
\end{itemize}
\end{ej}



\begin{ej}
    Demuestre que un subconjunto
    $W$ de $V$ es subespacio de $V$ si y sólo si coincide con el subespacio
    que genera en $V$, es decir, que
    \[
    W \leq V \Leftrightarrow
    W = \langle W \rangle
    \]
\end{ej}

\begin{ej}
    Si $X$ y $Y$ son subespacios1 de $V$, demuestre que
    \[
    \langle X \cup Y \rangle =
    \langle X \rangle + \langle Y \rangle
    =
    \langle \langle X \rangle \cup \langle Y \rangle \rangle
    \]
    Pista: para la segunda igualdad, simplemente use la definición de 
    suma de subespacios.
\end{ej}

\begin{ej}
($\mathbat$) De un ejemplo de un subconjunto no vacío
$U$ de $\IR^{2}$ que sea cerrado bajo multiplicación escalar,
pero que no sea subespacio de $\IR^{2}$.
\end{ej}

\begin{ej}
($\mathbat$) Si $W$ es un subespacio de $V$, describa
al subespacio $W + W$.
\end{ej}

\begin{ej}
    \label{ej: suma directa sii unicidad de representación del cero}
    Si $\{ W_{i} \}_{i=1}^{n}$ es una familia de subespacios de $V$,
    demuestre que las siguientes proposiciones son equivalentes:
    \begin{itemize}
        \item La suma $\sum_{i=1}^{n} W_{i}$ es directa.
        \item Si los vectores $w_{i} \in W_{i}$ ($1 \leq i \leq n$)
        son tales que $w_{1} + \cdots + w_{n} = 0$, entonces 
        toda $w_{i}$ es cero.
    \end{itemize}
    Es decir, demuestre que determinar si representaciones de la forma
    $w_{1} + \cdots + w_{n}$ con $w_{i} \in W_{i}$ son únicas equivale
    a ver la unicidad de la representación sólo del vector cero como suma
    de elementos de $\cup W_{i}$.
\end{ej}

\begin{ej}
	\label{ej: de la suma directa}
    Use a los subespacios
    \[
    W_{1} = \{ (x, y, 0) \in  \IR^{3} : \hspace{0.2cm} x, y \in \IR \},
    \]
    \[
    W_{2} = \{ (0, 0, x) \in  \IR^{3} : \hspace{0.2cm} x \in \IR \},
    \]
    \[
    W_{1} = \{ (0, x, x) \in  \IR^{3} : \hspace{0.2cm} x \in \IR \}
    \]
    de $\IR^{3}$ para dar un ejemplo de una familia de subespacios para la
    que, a pesar de que la intersección es $\{0\}$, la suma no es directa.
    Pista: para facilitarle el trabajo, use el ejercicio 
    \ref{ej: suma directa sii unicidad de representación del cero}.
\end{ej}

\begin{ej}
Si
\[
W_{1} = \{ (a_{1}, \ldots , a_{n}) \in F^{n}  | \hspace{0.2cm} 
a_{1} = 0 \},
\]
\[
W_{2} = \{ (a_{1}, \ldots , a_{n}) \in F^{n}  | \hspace{0.2cm} 
a_{2} = \cdots = a_{n} = 0 \},
\]
demuestre que $F^{n} = W_{1} \oplus W_{2}$.
\end{ej}


\begin{ej}
Sean $X, Y \subseteq V$. Demuestre que
\begin{itemize}
	\item Si $X \subseteq Y$ entonces $\langle X \rangle \leq 
	\langle Y \rangle$.
	\item Si $X \subseteq Y$ y $\langle X \rangle = V$,
	entonces $\langle Y \rangle = V$.
	\item $\langle X \cap Y \rangle \subseteq \langle X 
	\rangle \cap \langle Y \rangle$. Busque un ejemplo en el que
	se de la igualdad y otro en el que la contención sea propia.
\end{itemize}
\end{ej}

\begin{ej} ($\mathbat$)
Sea $b \in \IR$. Defina a la familia
\[
\mathcal{I}_{b} := \left\{
f \in \IR^{[0, 1]}: \hspace{0.2cm}
\int_{0}^{1} f(x) dx = b
\right\}.
\]
Demuestre que $\mathcal{I}_{b}$ es subespacio de $\IR^{[0,1]}$
si y sólo si $b = 0$.
\end{ej}

\begin{ej}
($\mathbat$)
Sea $A \neq \emptyset$ un conjunto cualquiera, $\IZ_{2} = \{ 0, 1 \}$
el campo de los enteros módulo $2$.

Podemos identificar a todo 
subconjunto $B \subseteq A$ con su \textbf{función característica},
es decir, la función $\chi_{B} : A \longrightarrow \IZ_{2}$ definida como
\begin{align*}
 \chi_{B}(x)= \begin{cases}
 1 & \textit{ si } x \in B, \\
 0 & \textit{ si } x \not\in B;
 \end{cases}
 \end{align*}
recíprocamente, toda función $f : A \longrightarrow \IZ_{2}$ 
induce un subconjunto de $A$ como 
\[
B = \{ x \in A : \hspace{0.2cm} f(x) = 1 \}.
\] 
Note que estos procesos son uno el inverso del otro, y que ellos
nos permiten establecer una biyección entre $\mathcal{P}(A)$
-el conjunto potencia de $A$- y el conjunto $\IZ_{2}^{A}$
de funciones de $A$ en $\IZ_{2}$. En la sección 
\ref{subsection: espacios vect de funciones}
vimos cómo dotar a $\IZ_{2}^{A}$ de estructura de 
$\IZ_{2}-$espacio vectorial.

Dados $\alpha \in \IZ_{2}$,
$f, g \in \IZ_{2}^{A}$, si $B$ y $C$ son los subconjuntos
de $A$ que estas funciones inducen,
\begin{itemize}
	\item ¿Qué subconjunto de $A$ induce la función suma $f+g$?
	\item ¿Cuál es el subconjunto de $A$ inducido por la función
	cero $\hat{0} : A \longrightarrow \IZ_{2}$?
	\item ¿Qué subconjunto de $A$ induce $-f$?
	\item ¿Cuál induce $\alpha f$?
	\item Si $\{ B_{i} \}_{i \in I}$ es una familia de subespacios
	de $A$ tal que $\{ \chi_{B_{i}}  | \hspace{0.2cm} 
	i \in I \}$ genera al $\IZ_{2}-$espacio vectorial 
	$\IZ_{2}^{A}$, demuestre que $A = \bigcup_{i \in I} B_{i}$.
\end{itemize}
  
\end{ej}