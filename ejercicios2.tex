\section{Ejercicios II}
En lo que sigue, a menos que se indique lo contrario,
$V$ es un $F-$espacio vectorial.
Los ejercicios marcados con el símbolo
$\mathbat$ son obligatorios para los estudiantes de matemáticas,
pero opcionales para los de actuaría.

\begin{ej}
Demuestre que si $f \in P_{n}(f)$ es tal que
$f(c_{j})=0$ para $n+1$ elementos distintos
$c_{0}, \ldots , c_{n}$ del campo $F$, entonces
$f$ es el polinomio cero. 
\textit{Pista:} use polinomios de interpolación de Lagrange
\end{ej}


\begin{ej}
Usando el teorema 
\ref{teo: la dim de subespacios es menor o igual a la del espacio}, 
demuestra que
\begin{itemize}
	\item los únicos subespacios de $\IR^{2}$ son $\{ 0 \}$,
	rectas que pasan por el origen y $\IR^{2}$, y
	\item los únicos subespacios de $\IR^{3}$ son 
	$\{ 0 \}$, rectas y planos que pasan por el origen,
	y $\IR^{3}$.
\end{itemize}
\end{ej}

\newpage