\section{Simulacro de examen, segundo parcial}

\begin{center}
	\textbf{Álgebra Lineal I, Otoño 2024 
	\hspace{0.1cm}	
	\S 
	\hspace{0.1cm}	
	Segundo examen parcial 
	\hspace{0.1cm}
	\S
	\hspace{0.1cm} 
	Lic. Amélie Bernès}
\end{center}



Es obligatorio que contestes los siguientes dos problemas; si tu 
respuesta es correcta, ganas los puntos correspondientes, pero si no los
contestas o tu respuesta es incorrecta, pierdes los puntos correspondientes.

\begin{prob}
($+$0.5/ $-$0.25) Enuncia la definición de transformación lineal.
\end{prob}

\begin{prob}
($+$0.5/$-$0.25) Si $T: V \longrightarrow W$ es una transformación lineal y 
$X \subseteq V$, 
$Y \subseteq W$, escribe las
definiciones de los conjuntos $T(X)$
y $T^{-1}(Y)$.
\end{prob}

\noindent\rule{\textwidth}{1pt}

Contesta los problemas que quieras de la siguiente lista. 
En los ejercicios numéricos, es recomendable que escribas limpiamente
tu procedimiento, para evaluar tu procedimiento en caso de que
la respuesta numérica final sea incorrecta.


\begin{prob}
(1 p)
Enuncia el teorema fundamental de las bases.
\end{prob}


\begin{prob}
(3 p.)
Enuncia y demuestra el teorema de la dimensión.
\end{prob}


\begin{prob}
(0.5 p) Sean $V$ y $W$ dos $F-$espacios vectoriales finito dimensionales,
$T: V\longrightarrow W$ una transformación lineal entre ellos.
Si $dim(V) < dim(W)$, explica por qué $T$ no puede ser suprayectiva.
\end{prob}


\begin{prob}
(1 p) 
Sea $T: V \longrightarrow W$ lineal. Demuestra que $T$ 
es inyectiva si y sólo si $Ker(T) = \{ 0_{V} \}$.
\end{prob}


\begin{prob}
(1.5 p)
Sea $T: V \longrightarrow W$ lineal. Demuestra que, si 
$\beta = \{ v_{1}, \ldots v_{n} \}$ es un subconjunto de $V$
tal que $T(\beta) \subseteq W$
es l.i., entonces $\beta$ también es l.i..
\end{prob}



\begin{prob}
(2 p) Sean 
$
\beta = \{ (1, 0, 0), (0, 1, 0), (0, 0, 1) \}\subseteq \IR^{3}$,
$
\gamma = \{ (3, 0), (1, -1) \} \subseteq \IR^{2} $
bases respectivas de $\IR^{3}$ y $\IR^{2}$.
Da la fórmula de 
la única transformación lineal $T: \IR^{3} \longrightarrow \IR^{2}$
tal que 
$$
[T]_{\beta}^{\gamma} = ·\begin{pmatrix}
1 & 7 & 0 \\
2 & 3 & 4
\end{pmatrix}.
$$
\end{prob}

\begin{prob}
Sean 
$$
\beta =
 \left\{
E_{1}= \begin{pmatrix}
1 & 0 \\
0 & 0
\end{pmatrix}, \hspace{0.2cm}
E_{2} = \begin{pmatrix}
0 & 1 \\
0 & 0
\end{pmatrix}, \hspace{0.2cm}
E_{3}= \begin{pmatrix}
0 & 0 \\
1 & 0
\end{pmatrix}, \hspace{0.2cm}
E_{4} = \begin{pmatrix}
0 & 0 \\
0 & 1
\end{pmatrix}
\right\} \subseteq M_{2 \times 2}(\IR),
$$
$$
\beta' =
 \left\{
A= \begin{pmatrix}
1 & 0 \\
0 & 2
\end{pmatrix}, \hspace{0.2cm}
B= \begin{pmatrix}
0 & 2 \\
3 & 0
\end{pmatrix}, \hspace{0.2cm}
C= \begin{pmatrix}
-1 & 0 \\
0 & 1
\end{pmatrix}, \hspace{0.2cm}
D= \begin{pmatrix}
0 & 3 \\
2 & 0
\end{pmatrix}
\right\} \subseteq M_{2 \times 2}(\IR),
$$
$$
\gamma = \{ 1 \} \subseteq \IR
$$
bases de $M_{2 \times 2}(\IR)$ y $\IR$, respectivamente.
Defínase $T: M_{2 \times 2}(\IR) \longrightarrow \IR$ como
$
T \left( \begin{pmatrix}
a & b \\
c & d
\end{pmatrix} \right) = a + d.
$
\begin{itemize}
	\item (1 p) Demuestra que $T$ es una transformación lineal.
	\item (1 p) Calcula $[T]_{\beta'}^{\gamma}$ a partir de la matriz
	$[T]_{\beta}^{\gamma}$ y una matriz de cambio de base.
	\item (1 p) Calcula el Kernel de $T$, su dimensión
	y la dimensión de la imagen de $T$.
\end{itemize}
\end{prob}

\begin{prob}
(1.5 p)
Considera a la transformación lineal $T: M_{2 \times 2}(\IR) 
\longrightarrow \IR^{4}$ dada por
\[
T 
\left(
\begin{pmatrix}
a & b \\
c & d
\end{pmatrix}
\right)
= (4a+3b+10c+d, 4a+b-2c-d, 8a-b+8c+d, -3b-8c-d).
\]
¿Es $T$ un isomorfismo? Argumenta tu respuesta.
\end{prob}