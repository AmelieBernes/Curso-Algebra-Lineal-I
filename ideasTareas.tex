\documentclass[10pt]{extreport}
\usepackage[a4paper, total={6in, 8in}]{geometry}
%\input{paquetes}

% ----------------------- Packages --------------------------

\usepackage[utf8]{inputenc}



\usepackage{babel}
%\decimalpoint
\usepackage{mathtools}% http://ctan.org/pkg/mathtool
\usepackage{amsthm, amsmath, bm, amssymb} %math packages
\usepackage{ wasysym, stmaryrd } %Para rayo de contradicción
\usepackage{lineno}

\usepackage{halloweenmath}
\usepackage{MnSymbol}

\usepackage{tikz}
\tikzset{every picture/.style={line width=0.75pt}} %set default line width to 0.75pt
\usepackage{imakeidx}
\makeindex


\usepackage[dvipsnames]{xcolor}
\newcommand{\mathcolorbox}[2]{\colorbox{#1}{$\displaystyle #2$}}

% --------------- highlight -----------------------------

\usepackage{soul}
\usepackage{hyperref}


\newcommand{\hlpink}[1]{{\sethlcolor{lime}\hl{#1}}}
\newcommand{\hlgray}[1]{{\sethlcolor{lightgray}\hl{#1}}}
\newcommand{\hllilac}[1]{{\sethlcolor{Orchid}\hl{#1}}}

% ---------------------------------------------------------

\usepackage{subcaption, threeparttable}
\usepackage{graphicx}
\graphicspath{{./imagenes}}
\DeclareCaptionFormat{custom}
{
	\textbf{#1#2}\textit{\small #3}
}
\captionsetup{format=custom}
\usepackage[font=small,labelfont=bf]{caption} %para que los nombres 'Figura x' estén en negritas.

%Para rotar imágenes:
\usepackage{wrapfig}
\usepackage{lscape}
\usepackage{rotating}
\usepackage{epstopdf}

\usepackage{marginfix}
\usepackage{marginnote}
\renewcommand*{\marginfont}{\footnotesize} %Para cambiar el tamaño de fuente de las marginnote: https://tex.stackexchange.com/questions/30473/specifying-font-size-in-a-newcommand



%Paquetes usados en estilo.tex
\usepackage{geometry}
%\usepackage{sidenotes}
%\usepackage[font=footnotesize,format=plain,labelfont={bf,sf},textfont={it},width=10pt]{caption}
% Captions at the side of the page
%\usepackage[wide]{sidecap}
\usepackage{morefloats}


\usepackage[backend=bibtex,style=alphabetic]{biblatex}
%\bibliography{referencias.bib}
\addbibresource{referencias.bib}
\usepackage{subfiles} % Best loaded last in the preamble


% ----------------------------------------------------------------------- QUOTES
\usepackage{csquotes}
\def\signed #1{{\leavevmode\unskip\nobreak\hfil\penalty50\hskip2em
		\hbox{}\nobreak\hfil(#1)% <-- edit this to change the looks of the author to e.g. "...\hfil - #1%" to get a similar output as in Skillmon's example (also, the %-sign needs to be there!)
		\parfillskip=0pt \finalhyphendemerits=0 \endgraf}}

\newsavebox\mybox
\newenvironment{signquote}[1]
{\savebox\mybox{#1}\begin{quote}}
	{\signed{\usebox\mybox}\end{quote}}
% -------------------------------------------------------------------



\usepackage{float}

\newcommand*{\IR}{\mathbb{R}}
\newcommand*{\IC}{\mathbb{C}}
\newcommand*{\IN}{\mathbb{N}}
\newcommand*{\IZ}{\mathbb{Z}}
\newcommand*{\IF}{\mathbb{F}}
\newcommand*{\IQ}{\mathbb{Q}}
\newcommand*{\QEDB}{\null\nobreak\hfill\ensuremath{\square}}%
\newcommand*{\diam}{\null\nobreak\hfill\ensuremath{\diamond}}%

\newcommand{\TODO}[1]{\textcolor{violet}{#1}} % TODO quitar. Sólo lo uso mientras estoy editando para resaltar en morado.

%Ambientes de teorema antiguos. TODO quitar esto.
\newtheorem{teo}{Teorema}[section]
%Pongo a todas el contador de 'teo'
\newtheorem{lema}[teo]{Lema}
\newtheorem{prop}[teo]{Proposición}
\newtheorem{obs}[teo]{Observación}
\newtheorem{cor}[teo]{Corolario}
\newtheorem{defi}{Definición}
\newtheorem{notacion}[teo]{Notación}
\newtheorem{ejem}[teo]{Ejemplo}
\newtheorem{prob}{Problema}
\newtheorem{ej}{Ejercicio}

%\input{estilo}
\usepackage{tcolorbox}
\tcbuselibrary{listingsutf8}

% Definir cuadro de ancho del texto
\NewTColorBox{boxProblem}{O{sidebyside=false, lower separated = true} m D(){#2}}{
	colback=purple!5!white,
	colframe=violet,
	colupper=violet!50!black,
	fontupper=\bfseries,
	fonttitle=\bfseries,
	label = {problem #3},
	title={#2},
	#1
}
%%%%%%%%%%%%%%%%%%%%%%%%%%%%%%%%%%%%%%%%%%%%%%%%%%%%%%%%%%%%%%%%%%%%%%%%%%%%


\begin{document}

\chapter{Ideas de problemas para tarea o examen}
		\begin{itemize}
		\item Demuestra que, si $\{ U_{\alpha} \}_{\alpha \in I}$ es una familia
		de subespacios de $V$, entonces $\cup U_{\alpha}$ contiene al vector cero
		y es cerrada bajo multiplicaciones escalares. Da un ejemplo en el que
		$\cup U_{\alpha}$ no sea subespacio de $V$.
		
		\item Si $U, V, W$ son subespacios de un $F-$espacio vectorial
		tales que $U \oplus  V = U \oplus W$, ¿necesariamente
		se tiene que $V = W$? Si la respuesta es afirmativa, demuéstrelo;
		si es negativa, de un contraejemplo.
		
		\item ¿El krank de una matriz es preservado bajo operaciones elementales
		de matrices?
		
		\item Si $V$ es un $F-$espacio vectorial, demuestre que $W \subseteq V$
		es subespacio de $V$ si y sólo si 
		\begin{itemize}
			\item $W \neq \emptyset$
			\item $(\forall x, y \in W)$ $(\forall \lambda \in F)$: 
			$\lambda x + y \in W$
		\end{itemize} 
		
		\item que te demuestren que $M(A)$ es un segmento inicial de los naturales
		(como lo pones en la tarea de abajo).
		
		\item Sea $C$ una colección l.d.. Se define el spark como
		... demuestra que si
		entonces es la única con la propiedad.
		Pista: la función $|| \cdot ||_{0}$ desigualdad triangular.
		
		\item Que completen la fórmula $dim(W + V)$. Que la usen para
		demostrar que dos subespacios de $\IR^{3}$ tienen suma directa, y que
		su suma es todo el espacio.
	\end{itemize}
	
	\newpage




 
\section{Rango y rango Kruskal de una matriz: Tarea 1.1}
\begin{flushright}
	\textbf{Amélie Bernès, Primavera 2025.}
\end{flushright}


En tu curso de teoría de ecuaciones ya trataste con el concepto de 
``rango'' de una matriz. A continuación, damos una posible definición.

\begin{defi} (c.f. \cite{rank})
	Si $A \in M_{m \times n}(\IR)$, el \textbf{rango} de $A$ 
	(denotado como $rank(A)$)
	es
	la mayor cantidad de columnas linealmente independientes de $A$.
\end{defi}

Si por $A_{j}$ denotamos al $j-$ésimo vector columna de $A$,
\begin{equation}
	\label{eq: matriz A con columnas}
	A = [A_{1} | A_{2} | A_{n} | \cdots | A_{n}], \hspace{0.2cm}
	A_{j} \in \IR^{m},
\end{equation}
para calcular el rango de la matriz $A$, se busca extraer un subconjunto linealmente
independiente de 
\[
Col(A) = \{ A_{1}, \ldots , A_{n} \} \subseteq \IR^{m}
\]
de cardinalidad máxima. Entonces, $rank(A) \in \IN$ es tal que 
\begin{itemize}
	\item existe un subconjunto de $Col(A)$ de cardinalidad $rank(A)$ que es
	linealmente independiente, y 
	\item todo subconjunto de $Col(A)$ de cardinalidad mayor a $rank(A)$ 
	es linealmente dependiente.
\end{itemize}



\begin{defi}
	(c.f. \cite{highDimDA}, p. 49)
	Si $A \in M_{m \times n}(\IR)$, el \textbf{rango Kruskal} de $A$ 
	(denotado como $krank(A)$)
	es el mayor número
	$r \in \IN$ tal que \textit{todo} subconjunto de $r$ 
	columnas de $A$ es linealmente independiente.
\end{defi}
O sea, para calcular el rango Kruskal de la matriz
\eqref{eq: matriz A con columnas}, debes considerar al subconjunto de los naturales
\[
M(A) = \{ r \in \IN \cup \{0\} : \hspace{0.15cm} \textit{ todo subconjunto de } Col(A) \textit{ de
r elementos es l.i..} \}
\]
y calcular su máximo;
\[
krank(A) = max (M(A)).
\]

Nota que $M(A)$ es un conjunto de números naturales consecutivos;
\[
M(A) = \{ 0, 1, \ldots, krank(A) \}.
\]
\begin{prob}
Si 
\[
A = \begin{pmatrix}
	1 & 0 & 0 \\
	2 & 1 & 2 \\
	3 & 2 & 4
\end{pmatrix},
\]
calcula $rank(A)$ y $krank(A)$.	
\end{prob}
\textbf{Solución:}
Sean $A_{1}, A_{2}, A_{3}$ las columnas de $A$.
Claro que 
\begin{equation}
		\label{eq: A2 multiplo A3}
		A_{3} = 2 A_{2}.
\end{equation}
\begin{itemize}
	\item Por \eqref{eq: A2 multiplo A3}, el rango de $A$ no puede ser $3$
	(el único subconjunto de $Col(A)$ de tres elementos contiene a $A_{2}$
	y $A_{3}$). Como $A_{1}$ y $A_{2}$ son l.i., entonces 
	\[
	rank(A) = 2.
	\]
	\item Por \eqref{eq: A2 multiplo A3}, 
	\[
	krank(A) = 1,
	\]
	pues
	sí hay singuletes 
	de columnas l.i., pero hay un conjunto de dos columnas l.d..
\end{itemize}

\begin{prob}
Demuestra que, si $A \in M_{m \times n}(\IR)$ es una
matriz cualquiera, entonces
\[
0 \leq krank(A) \leq rank(A).
\]	
\end{prob}
\textbf{Solución:} Si $r = krank(A)$, entonces todo subconjunto de $Col(A)$
de cardinalidad $r$ es l.i., luego, un l.i. de cardinalidad maximal no puede tener 
menos de $r$ elementos.
\begin{defi}
	Si $x = (x_{k})_{k=1}^{n} \in \IR^{n}$ y 
	\[
	sop(x) = \{ j \in \{ 1, \ldots , n \} : \hspace{0.2cm} x(j) \neq 0 \},
	\]
	se define la \textbf{norma cero} de $x$ como
	\begin{equation}
		\label{eq: norma cero}
	||x||_{0} := | sop (x) |.
	\end{equation}
\end{defi}
Entonces, la norma cero de un vector es la cantidad de entradas no cero de este.
A pesar del nombre, la expresión \eqref{eq: norma cero} no define una norma
en $\IR^{n}$ (no se cumple la desigualdad triangluar). 
\begin{prob}
	\label{prob: 3 de Kruskal}
Demuestra que, para $A \in M_{m \times n}(\IR)$, son equivalentes
\begin{enumerate}
	\item $krank(A) \geq k$
	\item El único vector $x \in \IR^{n}$ tal que $Ax = 0$ y $||x||_{0} \leq k$ es
	el vector cero. 
\end{enumerate}
Sugerencia: Utiliza la ecuación

\begin{equation}
	\label{eq: Ax comb lineal de columnas}
	Ax = x_{1} A_{1} + \ldots + x_{n} A_{n} \in \IR^{m}.
\end{equation}

	
\end{prob}
\textbf{Solución:} 
	
	Claro que $krank(A) \geq k$ si y sólo si todo subconjunto
	de $Col(A)$ de $k$ elementos es l.i.. (\hlgray{poner en el examen?})
	\begin{itemize}
	\item [$\Rightarrow$)]
	\TODO{pon aparte el caso k igual a cero?}
	Supongamos que existe $x = (x_{k})_{k=0}^{n-1}$ un vector
	no cero tal que $Ax = 0$ y $||x||_{0} = l \leq k$. Sin pérdida de
	generalidad, digamos que 
	\[
	x = (x_{1}, \ldots , x_{l}, 0, \ldots , 0),
	\]
	con $x_{1}, \ldots, x_{l} \neq 0$.
	Usando el que $Ax = 0$ y la relación \eqref{eq: Ax comb lineal de columnas},
	se tiene que
	\[
	0 = Ax = x_{1} A_{1} + \ldots + x_{l} A_{l}, 
	\hspace{0.2cm} x_{1}, \ldots, x_{l} \neq 0;
	\]
	esto contradice el que $\{ A_{1}, \ldots , A_{l} \}$ sea l.i..
	
	\item[$\Leftarrow$)] Mostremos que todo subconjunto de $k$
	columnas de $A$ es l.i.. Para simplificar la notación, tomemos a 
	$A_{1}, \ldots , A_{k}$. Sean $x_{1}, \ldots , x_{k}$ escalares 
	tales que
	\[
	0 = x_{1} A_{1} + \ldots + x_{k} A_{k}.
	\]
	Si definimos al vector 
	\[
	x = (x_{1}, \ldots , x_{k}, 0, \ldots , 0) \in \IR^{n},
	\]
	por la ecuación anterior tenemos que $x$ es solución de
	la ecuación $Ax = 0$. Además, $|| x ||_{0} \leq k$, pues las únicas
	entradas de $x$ que pueden no ser cero son $x{1}, \ldots , x_{k}$.
	Así, por hipótesis, $x$ es el vector cero, luego, 
	$x_{1} = \cdots = x_{k} = 0$. De esto concluimos que 
	$\{A_{1}, \ldots , A_{k}\}$ es l.i..
\end{itemize}


\vspace{1cm}

\[
\diamond \diamond \diamond
\]

\vspace{1cm}

La equivalencia establecida en el Problema 
\ref{prob: 3 de Kruskal} se usa para determinar unicidad de soluciones
en problemas de optimización del tipo

\begin{align*}
	\textit{minimizar} & \hspace{0.3cm} || x ||_{0}, \\
	\textit{donde} & \hspace{0.3cm} Ax = y.
\end{align*}
Véase, por ejemplo, \cite{highDimDA}, Teorema 2.6, p. 49.

{\let\clearpage\relax \printbibliography}


\newpage

\section{Tarea 1.2}
\begin{flushright}
	\textbf{Amélie Bernès, Primavera 2025.}
\end{flushright}


\begin{prob}
	Sean $X \neq \emptyset$ un conjunto no vacío, $F$ un campo. 
	Considere al
	$F-$espacio vectorial 
	$F^X$ de funciones de $X$ en $F$ (como se definió en clase).
	Seleccione un punto $x_{0} \in X$ y defina a $W$
	como el conjunto de funciones de $X$ en $F$ que mapean el punto
	$x_{0}$ al cero del campo, es decir, 
	\[
	 W = \{ f \in F^X : \hspace{0.2cm} f(x_{0}) = 0 \} \subseteq F^X.
	\]
	Demuestra que $W$ es subespacio de $F^X$.
\end{prob}


\begin{prob}
	Considere al $\IR-$espacio vectorial $\IR^{3}$.
	Si
	\[
	 X = \{ v_{1}, v_{2}, v_{3} \}, 
	\]
	donde
	\[
	v_{1} = (1, 1, 1), v_{2} = (0, 1, -1), v_{3} = (1, 0, 2),
	\]
	demuestra que el vector 
	\[
	u = (4, 2, 1) \in \IR^{3}
	\]
	\textbf{no} es elemento del generado de $X$.
\end{prob}

\end{document}